\documentclass{article}
\usepackage[utf8]{inputenc}

\title{CS7643 Group (our group's name) Project Proposal}
\author{Megan Aldridge, Colleen Giannotta, Sankaranarayanan Ramamoorthy, Ashley Towne}
\date{February 2022}

\begin{document}

\maketitle

\section{Project Title: Summary}

Project summary (4-5+ sentences). Fill in your problem and background/motivation (why do you want to solve it? Why is it interesting?). This should provide some detail (don’t just say “I’ll  be working on object detection”)

\section{Approach}
What you will do (Approach, 4-5+ sentences) - Be specific about what you will implement and what existing code you will use. Describe what you actually plan to implement or the experiments you might try, etc. Again, provide sufficient information describing exactly what you’ll do. One of the key things to note is that just downloading code and running it on a dataset is not sufficient for a description or a project! Some thorough implementation, analysis, theory, etc. have to be done for the project.

\section{Resources/Related Work & Papers}
Resources / Related Work & Papers (4-5+ sentences). What is the state of art for this problem? Note that it is perfectly fine for this project to implement approaches that already exist. This part should show you’ve done some research about what approaches exist.

\begin{enumerate}
    \item First article idea
    \item Megan's article
\end{enumerate}

\section{Datasets}
Datasets (Provide a link to the dataset). This is crucial! Deep learning is data-driven, so what datasets you use is crucial. One of the key things is to make sure you don’t try to create and especially annotate your own data! Otherwise, the project will be taken over by this.

\begin{enumerate}
    \item First dataset idea
    \item http://pic2recipe.csail.mit.edu/
\end{enumerate}

\section{Group Members}
Megan Aldridge
Colleen Giannotta
Sankar Ramamoorthy
Ashley Towne

\end{document}
